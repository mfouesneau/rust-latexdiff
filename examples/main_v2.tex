\documentclass{article}
\usepackage{xcolor}
\usepackage[normalem]{ulem}


\newcommand{\DIFdel}[1]{\textcolor{red}{\sout{#1}}}
\newcommand{\DIFdelbegin}{\textcolor{red}{\bgroup\sout\bgroup}}
\newcommand{\DIFdelend}{\egroup\egroup}


\newcommand{\DIFadd}[1]{\textcolor{blue}{#1}}
\newcommand{\DIFaddbegin}{\textcolor{blue}{\bgroup}}
\newcommand{\DIFaddend}{\egroup}

\title{Example Document}
\author{Test Author}

\begin{document}
\maketitle

% BEGIN INCLUDED FILE: chapter1_v2.tex
\section{Introduction}

This is the first chapter of our document. It contains some basic text
and demonstrates the functionality of the LaTeX expansion tool.

\subsection{Background}

Here we discuss the background of our work. This section has been significantly
expanded in this version of the document with new research findings.


\begin{equation}
E = mc^2
\end{equation}

The famous equation above shows the relationship between energy and mass.
We have also discovered that this relationship has important implications
for our research.
% END INCLUDED FILE: chapter1_v2.tex

% BEGIN INCLUDED FILE: chapter2.tex
\section{Methods}

This chapter describes the methods used in our research.

\subsection{Experimental Setup}

We used the following equipment:
\begin{itemize}
\item Microscope
\item Computer
\item Data acquisition system
\end{itemize}

\subsection{Data Analysis}

The data was analyzed using statistical methods and specialized software.
% END INCLUDED FILE: chapter2.tex

\end{document}
